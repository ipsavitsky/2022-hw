\documentclass{article}
\usepackage[utf8]{inputenc}
\usepackage[T2A]{fontenc}
\usepackage[russian]{babel}
\usepackage[
backend=biber,
style=gost-numeric,
% sorting=ynt
]{biblatex}

\addbibresource{references.bib}

\title{Формальная постановка задачи второй задачи практикума}
\author{Савицкий Илья, 421 группа}
\date{}

\begin{document}
\maketitle
\section*{Дано}
\begin{enumerate}
    \item Вычислительная система, состоящая из $p$ однородных процессоров
    \item Множество $N$, состоящее из $n$ независимых работ, для каждой работы задано время ее выполнения.
    \item Вектор $T$ длины $n$ времен выполнения работ на процессорах. Элемент $N_i$ равен времени выполнения $i$-й работы. 
\end{enumerate}
\section*{Требуется}
\begin{enumerate}
    \item Построить расписание $HP$, то есть для $i$-й работы определить время начала выполнения $s_i$ и процессор $p_i$, на котором она будет выполняться
\end{enumerate}
\subsection*{Определение расписания}
Расписание оопределено, если заданы:
\begin{enumerate}
    \item Множества процессоров и работ
    \item Привязка - всюду определенная на множестве работ функция, которая задает распределение работ на процессорах
    \item Порядок - для каждого процессора определен порядок выполнения работ на конкретном процессоре.
\end{enumerate}
Это определение соответствует графической форме представления расписания (представление исключительно в виде привязки и порядка), однако в нашей задаче будет строится временная диаграмма, поскольку нам полезно получить результат в виде определенных пар $\left( s_i, p_i \right)$. Доказано, что эти формы представления эквивалентны \cite{Kalashnikov_2004}.
\subsection*{Ограничение на корректность расписания}
    \begin{enumerate}
        \item Каждый процессор за единицу времени может выполнять не больше одной работы.
        \item Прерывание работ недопустимо, перенос частично выполненной работы на другой процессор недопустим.
    \end{enumerate}
\section*{Минимизируемый критерий (K1)}
Требуется минимизировать длительность расписания, то есть выбрать такую работу $k$, для которой $s_k + N_k \rightarrow \max$, где $s_k$ - время начала выполнения работы на процессоре, а $N_k$ - время выполнения работы на процессоре.  
\printbibliography
\end{document}